\documentclass[a4paper]{exam}

\usepackage{amsmath}
\usepackage{geometry}
\usepackage{graphicx}
\usepackage{hyperref}
\usepackage{titling}

% Header and footer.
\pagestyle{headandfoot}
\runningheadrule
\runningfootrule
\runningheader{CS 212, Fall 2022}{WC 08: Decidability}{\theauthor}
\runningfooter{}{Page \thepage\ of \numpages}{}
\firstpageheader{}{}{}

\printanswers

\title{Weekly Challenge 08: Decidability}
\author{ungraded} % <=== replace with your student ID, e.g. xy012345
\date{CS 212 Nature of Computation\\Habib University\\Fall 2022}

\qformat{{\large\bf \thequestion. \thequestiontitle}\hfill}
\boxedpoints

\begin{document}
\maketitle

\begin{questions}
  
\titledquestion{Deciding Palindromes}

  Show that determining whether a given string over $\{0,1\}$ is a palindrome is a decidable problem.

You may proceed as follows. All references below are from the textbook.
\begin{parts}
\part Formulate the problem in terms of membership in a language. For example, see the languages $D$ and $D_1$ in the subsection ``Hilbert’s Problems'' on Page 182.
\part Show that this language is Turing-decidable (see Definition 3.6) by giving a Turing machine algorithm that decides it. You may provide \textit{formal}, \textit{implementation}, or \textit{high-level} descriptions of the algorithm (see ``Terminology for describing Turing Machines'' on Page 184). For example, formal descriptions are given in Examples 3.7 and 3.9, implementation descriptions are given in Examples 3.7, 3.11 and 3.12, and a high-level description is given in Example 3.23.
\end{parts}
  
  \begin{solution}
    % Enter your solution here.
  \end{solution}
\end{questions}
\end{document}

%%% Local Variables:
%%% mode: latex
%%% TeX-master: t
%%% End:
